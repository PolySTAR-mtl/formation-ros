\documentclass{beamer}
\usepackage{tikz}
\usepackage[french]{babel}
\usepackage[T1]{fontenc}
\usepackage{dirtree}
\usepackage{minted}
\usepackage{hyperref}

\usetheme{metropolis}
\setcounter{errorcontextlines}{999}
\DTsetlength{0.2em}{1em}{0.2em}{0.2pt}{1.6pt}
\setlength{\DTbaselineskip}{10pt}

\title{PolySTAR - Equipe CV\\Formation ROS}
\author{Sébastien Darche}
\institute{PolySTAR - Polytechnique Montréal}
\date{\today}

\begin{document}

\frame{\titlepage}


%% INTRODUCTION
\begin{frame}
\frametitle{Introduction}

Robot Operation System (ROS), framework logiciel pour la robotique

\begin{itemize}
    \item Producteur - consommateur communiquant par des topics
    \item "Services" (RPC) synchrones
    \item Serveur de paramètres
    \item Indépendant du langage utilisé (définition des messages), mais principalement Python et C++
    \item Potentiellement réparti
    \item Système de build (catkin)
    \item Système de lancement de configuration (roslaunch)
\end{itemize}

\end{frame}


%% Init cpp
\begin{frame}
\frametitle{Organisation workspace}
\footnotesize{
\dirtree{%
.1 \texttt{ros\_ws/}.
    .2 \texttt{.catkin\_workspace}.
    .2 \texttt{CMakeLists.txt}.
    .2 \textit{devel/}.
    .2 \textit{build/}.
    .2 \texttt{src/}.
        .3 \texttt{mon\_module\_cpp}.
            .4 \texttt{include/}.
            .4 \texttt{src/}.
            .4 \texttt{msg/}.
            .4 \texttt{launch/}.
            .4 \texttt{CMakeLists.txt}.
        .3 \texttt{mon\_module\_python}.
            .4 \texttt{script/}.
            .4 \texttt{msg/}.
            .4 \texttt{launch/}.
            .4 \texttt{CMakeLists.txt}.
}
}
\end{frame}

%% Init cpp
\begin{frame}[fragile]
\frametitle{Initialisation
C++\footnote{\href{http://wiki.ros.org/ROS/Tutorials/WritingPublisherSubscriber\%28c\%2B\%2B\%29
}{Tuto ROS C++}}
}
\begin{minted}[fontsize=\scriptsize]{cpp}
#include <ros/ros.h>
#include "mon_module_cpp/MonMessage.h"

int main(int argc, char** argv) {
    ros::init(argc, argv, "nom_noeud");
    ros::NodeHandle nh;

    ros::Publisher pub = 
        nh.advertise<mon_module_cpp::MonMessage>("topic1",1);

    ros::Subscriber sub = nh.subscribe("topic2", 1, callback);
    ros::spin();
}
\end{minted}

\end{frame}


%% Init python
\begin{frame}[fragile]
\frametitle{Initialisation Python\footnote{\href{http://wiki.ros.org/ROS/Tutorials/WritingPublisherSubscriber\%28python\%29
}{Tuto ROS Python}}}

\begin{minted}[fontsize=\scriptsize]{python}
import rospy

from mon_module_cpp.msg import MonMessage # Autre module
from std_msgs.msg import String

rospy.init_node('nom_noeud', anonymous=True)
pub = rospy.Publisher('topic1', MonMessage, queue_size=1)
sub = rospy.Subscriber('topic2', String, callback, queue_size=1)
rospy.spin()
\end{minted}
\end{frame}
    
%% Messages
\begin{frame}[fragile]
\frametitle{Définition des messages\footnote{\href{http://wiki.ros.org/ROS/Tutorials/CreatingMsgAndSrv
}{Tuto ROS msg/srv}}}
Ex : \texttt{mon\_module\_cpp/msg/MonMessage.msg}

\begin{minted}[fontsize=\scriptsize]{text}
float32 un_float            # Commentaire
float32[] tableau_float
String str
UnAutreMessage msg
time timestamp
sensor_msgs/Image img       # Image OpenCV
\end{minted}

\end{frame}


%% Catkin_make
\begin{frame}
\frametitle{\texttt{catkin\_make}}
\begin{itemize}
    \item Système de build basé sur CMake/Make
    \item Se lance à la racine du workspace : \texttt{ros\_ws}
\end{itemize}
\end{frame}

%% Lancer des noeuds
\begin{frame}
\frametitle{Lancer un/des noeuds}
Ne pas oublier de \texttt{source devel/setup.bash}
\begin{itemize}
    \item Automatisé : \begin{itemize}
        \item \texttt{roslaunch} \textit{fichier}.\texttt{launch}
        \item Syntaxe XML
        \item Possibilité d'inclure des sous-fichiers (slide suivante)
    \end{itemize}
    \item Manuel : \begin{itemize}
        \item Lancer un \texttt{roscore} : serveur de dispatch des messages ROS \footnote{ou
        alternativement donner une IP, env \texttt{ROS\_HOSTNAME} et \texttt{ROS\_MASTER\_URI},
        \href{http://wiki.ros.org/ROS/NetworkSetup}{Turo ROS Network}}
        \item \texttt{rosnode start} \textit{nodename} [\texttt{param1:=value1} ...]
    \end{itemize}
\end{itemize}
\end{frame}

%% roslaunch
\begin{frame}[fragile]
\frametitle{\texttt{roslaunch}\footnote{\href{http://wiki.ros.org/roslaunch/XML
}{Référence XML roslaunch}}}
Ex : \texttt{ros\_ws/tuto.launch}

\begin{minted}[fontsize=\scriptsize]{text}
<launch>
    <!-- Args -->
    <arg name="argument" default="False"/>

    <!-- Parameters -->
    <param name="/param_global" value="42" />

    <!-- Sub-files -->
    <include file="$(find mon_module_cpp)/launch/default.launch">
        <arg name="local" value="$(eval not arg('argument'))" />
    </include>

    <!-- Nodes -->
    <node name="nodepy" pkg="mon_module_python" type="nodepy"/>
</launch>
\end{minted}
\end{frame}


%% Utilitaires en terminal
\begin{frame}
\frametitle{Utilitaires}
\begin{itemize}
    \item \texttt{rosnode list} : liste des noeuds lancés
    \item \texttt{rostopic list} : liste des topics fournis
    \item \texttt{rostopic echo} : écrire les messages d'un topic
    \item \texttt{rosparam} \texttt{set} / \texttt{get} / \texttt{list} ..
\end{itemize}
\end{frame}

%% Utilitaires en terminal
\begin{frame}
\frametitle{En apprendre plus ..}
\begin{itemize}
    \item \texttt{rviz} : outil de visualisation 3D pour la robotique
    \item \texttt{rqt} : dashboard de visualisation capteurs / topics
    \item \texttt{tf2} : gestion des référentiels
    \item \texttt{dynamic\_reconfigure} : changer des paramètres à la volée
    \item \texttt{rosbag} : enregistrement / lecture de datasets
    \item ...
\end{itemize}
\end{frame}
    
\end{document}
